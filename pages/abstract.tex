\chapter{\abstractname}
Smart contracts, which are immutable programs that automate agreements between parties, are critical components in many blockchain networks. Their increasing role in sensitive financial transactions has heightened the importance of their security.
However, traditional software testing methodologies, like unit testing, have proven inadequate in improving smart contract security.
Instead, fuzzing has emerged as a promising technique for finding bugs in smart contracts.
While fuzzing techniques have faced challenges in adapting to smart contracts' distinct nature, property-based fuzzing tools such as Echidna have shown promise for Ethereum smart contracts.
With the emergence of newer blockchain platforms like Algorand, which offers advantages like high throughput, low latency, and lower transaction fees, there is a pressing need for specific tooling to ensure secure smart contracts.
This thesis introduces AlgoFuzz, a property-based fuzzing tool tailored for Algorand smart contracts.
The tool uses greybox fuzzing techniques to maximize code coverage and the discovered state space.
To evaluate the efficacy of AlgoFuzz, it is examined through experiments on adapted Algorand smart contracts from the Echidna benchmarks.
