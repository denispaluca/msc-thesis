% !TeX root = ../main.tex
% Add the above to each chapter to make compiling the PDF easier in some editors.

\chapter{Discussion}\label{chapter:discussion}
In this chapter we will discuss the findings of our study and the threats to the validity of the results.

\section{Findings and Interpretations}

\subsection*{RQ.1 Code Coverage}
The results in section \ref{section:results-code-coverage} show our approach was able to achieve a code coverage of 65\% on average.
The code coverage also when we exclude the coverage achieved by the seed values is substantial.
AlgoFuzz was able to achieve a code coverage of 32\% on average without the seed values.
Although 100\% code coverage was not achieved in any of the runs, we assume this is due to the fact that some lines of code are not reachable, for example lines that are executed during the deployment of the smart contract.

Our data indicate the usual trend with regard to code coverage wherein the code coverage increases rapidly initially, but then it tends to plateau as the fuzzing session is extended.
Because the fuzzer works with structured inputs, it does not have to surpass the parse of the program under test.
Therefore, during the initial phases of fuzzing, the most straightforward and easily reachable code paths are explored.
As most of the code paths are explored, the probability of finding new code paths diminishes.
This of course is dependent on the structure of the smart contract being tested.

\begin{mybox}
    \textbf{Answer to RQ.1:} The results of our study demonstrate that AlgoFuzz is proficient in covering new code.
    In most instances, AlgoFuzz was able to attain the maximal coverage rate regardless of the configuration.
    Even when accounting for the coverage gained by the seed values, AlgoFuzz was able to achieve additional coverage. The best configuration gained on average an additional 34\% coverage.
\end{mybox}

\subsection*{RQ.2 State Discovery}
When dealing with state discovery, we cannot simply create a ratio of the number of unique states to the total number of states.
This is because of the state explosion problem (or combinatorial explosion) which is a phenomenon where the number of possible states increases exponentially with the number of variables.
Considering this, the total number of states would be in most cases much larger than the discovered states bringing the ratio close to zero.

To understand if AlgoFuzz is able to effectively discover new states, we analyzed the ratio of unique transitions to the number of calls made to the smart contract in section \ref{section:results-state-discovery}.
Our data reveal that with the right configuration AlgoFuzz is able to discover new transitions very effectively.
As transitions and states are directly related, we can infer that AlgoFuzz is also able to discover new states effectively.

\begin{mybox}
    \textbf{Answer to RQ.2:} The results show that AlgoFuzz with the State driven configuration is able to produce 0.9 unique transitions per call on average.
    Although, there is no other tool to compare this result to, considering that the maximum ratio is 1.0, we can conclude that AlgoFuzz is able to discover new states effectively.
\end{mybox}

\section{Threats to Validity}
In this section we will discuss the threats to the validity of our study.
We differentiate between internal and external threats to validity.
Internal validity referring to the correctness of the drawn conclusions and minimization of methodological errors.
External validity refers to the extent the results can be generalized to other contexts.

\subsection*{Internal}
As we already mentioned in \ref{section:results}, we had to exclude runs from the analysis.
We assume that the LRZ server may have been under heavy load during the time of the runs.
Other runs might have also been affected by the load.
Although we currently have no way of knowing the impact of the load on the different runs, we cannot exclude that it had an impact on the results.

Another threat to internal validity is the non-deterministic behavior of the smart contract because of different blockchain states.
We run the smart contracts against the same local Algorand network without resetting the network, since doing so after each run would have been too time-consuming.
For this reason, the results of the runs may have been affected by not starting out with a clean state of the network.

\subsection*{External}
The external validity of our study is limited by the number of smart contracts that we analyzed and by the fact that we only analyzed smart contracts which were ported to Algorand by us.
AlgoFuzz cannot be used to analyze smart contracts which do not adhere to the official \ac{ABI}.
This is the case for the vast majority of the most popular smart contracts on Algorand \cite{noauthor_algorand_nodate-6}.
Other limitations such as lack of dynamic storage support (box storage) and lack of support for \acp{ASA} also affect the external validity of our study.

The contracts we analyzed may also not be representative of the smart contracts that are currently deployed on Algorand.
Contracts on Algorand may be more complex than the ones we analyzed.
Aside from this, these contracts may have bugs which were introduced during the porting process.

In future \ac{ABI} versions, more functionality may be added which may not be compatible with AlgoFuzz.
This may limit the external validity of our findings for contracts which are written with an \ac{ABI} version that is not currently supported.