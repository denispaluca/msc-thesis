% !TeX root = ../main.tex
% Add the above to each chapter to make compiling the PDF easier in some editors.

\chapter{Introduction}\label{chapter:introduction}
In recent years, blockchain technology has gained immense traction, revolutionizing various industries because of its decentralized and transparent nature.
The rapid growth of blockchain technology has brought a new era of decentralized applications, which are made possible by smart contracts.
Smart contracts are one of the key components of most modern blockchain networks.
They are self-executing programs that run on the blockchain and are used to automate the execution of agreements between two or more parties.
As smart contracts handle an increasing set of sensitive transactions, ranging from financial agreements to supply chain management, making them secure has become a top priority.
The financial value locked in these smart contracts coupled with the fact that they are immutable (meaning their code cannot be changed once deployed), makes exploiting their vulnerabilities a lucrative target for attackers.
In the recent past, we have witnessed several high-profile attacks on smart contracts, which have resulted in the loss of billions of dollars in value.

As with other software, different software testing techniques have been proposed and are being actively used to detect vulnerabilities in smart contracts.
Traditionally software testing methodologies such as unit testing, have shown to be minimally effective in improving the security of smart contracts when used on their own.
Unit tests are only as good as the test cases written by the developer, and it is impossible to write test cases for all possible inputs.
For this reason, fuzzing has been proposed as a complementary technique to unit testing.

Fuzz testing or fuzzing is a software testing technique that involves providing invalid, unexpected, random or semi-random data as inputs to a program.
The goal of fuzzing is to trigger unexpected behavior in the program, which may indicate the presence of a vulnerability.
During fuzzing the program is monitored for problems such as crashes, assertion failures, and memory leaks.
Most fuzzer require little to no knowledge of the program under test, making them easy to use.
Also, considering the automated nature of fuzzing, it is simple to integrate it into the software development lifecycle.
Fuzzing has achieved great success in detecting security flaws, and because of this, it has become the most widely used technique to discover vulnerabilities.

Fuzzing has been used to detect vulnerabilities for smart contracts as well.
However, adapting fuzzing techniques to smart contracts has presented its own set of challenges.
Different from normal programs, in smart contracts crashes, assertion failures and memory leaks are not a concern since the transaction will simply be reverted.
For this reason, most existing fuzzers for smart contracts focus on detecting some specific common vulnerabilities.
A different approach is to allow the user to specify a set of properties that the fuzzer should check for.
This approach is called property-based fuzzing.
The main drawback of this approach is that it requires the user to have a good understanding of the smart contract and the vulnerabilities that it may contain.
Despite this, different property-based fuzzing tools such as Echidna, have shown to be effective in detecting real-world vulnerabilities.

Similar to Echidna, most of the existing work on smart contract fuzzing has focused on the Ethereum blockchain since it is the most widely used blockchain for smart contracts.
Novel blockchain platforms that support smart contracts and which address the shortcomings of Ethereum have emerged in recent years.
One such platform is the Algorand blockchain.
With its unique \ac{PPoS} consensus mechanism, Algorand has been able to achieve high throughput and low latency while maintaining decentralization.
Different from Ethereum, Algorand provides immediate transaction finality, which means that once a block is added to the blockchain, it cannot be reverted and that no forks can occur.
Transaction fees in Algorand are also significantly lower than in Ethereum.
These features have made Algorand an attractive platform for smart contracts.
However, the lack of a mature tooling ecosystem for Algorand smart contracts has made it difficult for developers to write secure smart contracts.

In this thesis, we propose a novel fuzzing tool designed specifically for Algorand smart contracts called \textit{AlgoFuzz}.
AlgoFuzz is a property-based fuzzer with a feature set similar to Echidna.
Our tool addresses the challenges
The fuzzer leverages greybox fuzzing techniques to generate inputs that maximize not only code coverage but also the state space of the smart contract.
