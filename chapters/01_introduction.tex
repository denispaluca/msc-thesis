% !TeX root = ../main.tex
% Add the above to each chapter to make compiling the PDF easier in some editors.

\chapter{Introduction}\label{chapter:introduction}
In recent years, blockchain technology has gained immense traction, revolutionizing various industries because of its decentralized and transparent nature \cite{noauthor_global_nodate}.
The rapid growth of blockchain technology has brought a new era of \acp{dApp}, which are made possible by smart contracts.
Smart contracts are one of the key components of most modern blockchain networks.
They are self-executing programs that run on the blockchain and are used to automate the execution of agreements between two or more parties \cite{zou_smart_2021}.
As smart contracts handle an increasing set of sensitive transactions, ranging from financial agreements to supply chain management, making them secure has become a top priority.
The financial value locked in these smart contracts coupled with the fact that they are immutable (meaning their code cannot be changed once deployed), makes exploiting their vulnerabilities a lucrative target for attackers.
In the recent past, we have witnessed several high-profile attacks on smart contracts, which have resulted in the loss of billions of dollars in value  \cite{noauthor_funds_nodate}.
One such attack was the Wormhole hack, which resulted in the loss of \$320 million worth of cryptocurrency \cite{faife_wormhole_2022}.

As with regular software, different software testing techniques have been proposed and are being actively used to detect vulnerabilities in smart contracts.
Traditionally software testing methodologies such as unit testing, have shown to be minimally effective in improving the security of smart contracts when used on their own \cite{noauthor_smart_nodate}.
Unit tests are only as good as the test cases written by the developer, and it is impossible to write test cases for all possible inputs.
For this reason, fuzzing has been proposed as a complementary technique to unit testing.

Fuzz testing or fuzzing is a software testing technique that involves providing invalid, unexpected, random or semi-random data as inputs to a program \cite{manes_art_2019}.
The goal of fuzzing is to trigger unexpected behavior in the program, which may indicate the presence of a vulnerability.
During fuzzing the program is monitored for problems such as crashes, assertion failures, and memory leaks.
Most fuzzer require little to no knowledge of the program under test, making them easy to use.
Also, considering the automated nature of fuzzing, it is simple to integrate it into the software development lifecycle.
Fuzzing has achieved great success in detecting security flaws, and because of this, it has become the most widely used technique to discover vulnerabilities \cite{li_fuzzing_2018, zhu_fuzzing_2022}.

Fuzzing has been used to detect vulnerabilities for smart contracts as well.
However, adapting fuzzing techniques to smart contracts has presented its own set of challenges.
Smart contracts are distinct from traditional programs in that concerns like crashes, assertion failures, and memory leaks typically do not impact them. If such issues arise, the transaction is simply reverted, ensuring the blockchain's state remains unchanged.
For this reason, most existing fuzzers for smart contracts focus on detecting some specific common vulnerabilities, such as \textit{reentrancy} where a function call is exploited to re-run multiple times before finishing.
A different approach is to allow the user to specify a set of properties that the fuzzer should check for.
This approach is called property-based fuzzing.
It requires the user to have a good understanding of the smart contract and the vulnerabilities that it may contain.
Despite this, different property-based fuzzing tools such as Echidna, have shown to be effective in detecting real-world vulnerabilities \cite{grieco_echidna_2020, noauthor_echidna_nodate}.

Similar to Echidna, most of the existing work on smart contract fuzzing has focused on the Ethereum blockchain since it is the most widely used blockchain for smart contracts \cite{guo_analysis_2022}.
Novel blockchain platforms that support smart contracts and address the shortcomings of Ethereum have emerged in recent years.
One such platform is the Algorand blockchain \cite{chen_algorand_2019}.
It uses a unique consensus mechanism called \ac{PPoS}, where users are randomly and secretly selected to propose new blocks.
The likelihood that a user will be selected and the weight of its proposals and votes, are directly proportional to its stake.
With this consensus mechanism, Algorand has been able to achieve high throughput and low latency while maintaining decentralization.
Different from Ethereum, Algorand provides immediate transaction finality, which means that once a block is added to the blockchain, it cannot be reverted and that no forks can occur.
Transaction fees in Algorand are also significantly lower than in Ethereum.
These features have made Algorand an attractive platform for smart contracts.
However, the lack of a mature tooling ecosystem for Algorand smart contracts has made it difficult for developers to write secure smart contracts.
Static analyzers, such as Tealer \cite{noauthor_crytictealer_nodate} and Panda \cite{sun_panda_2023}, also testing frameworks, such as AlgoBuilder \cite{noauthor_algo_nodate}, have been introduced there is still a lack of fuzzing tools for Algorand smart contracts.

In this thesis, we propose a novel fuzzing tool designed specifically for Algorand smart contracts called \textit{AlgoFuzz}.
AlgoFuzz is a property-based fuzzer with a feature set similar to Echidna.
Our tool tests Algorand smart contracts in a dockerized environment running a local sandbox version of an Algorand node.
AlgoFuzz leverages \textit{greybox} fuzzing techniques to generate inputs that maximize not only code coverage but also the discovered state space of the smart contract.

The main goal of this project is the development of AlgoFuzz, which will serve as a prototype in the field of Algorand smart contract fuzzing.
We also evaluate the effectiveness of AlgoFuzz  through a series of empirical experiments.
The experiments are conducted on a set of Algorand smart contracts which are translated from the Ethereum smart contracts used in the Echidna benchmark suite.
We also evaluate the performance of AlgoFuzz on two larger Algorand smart contracts, one of which is translated from the USD Tether smart contract and the other is a smart contract for a token exchange.

\section*{Thesis Structure}
The remainder of this thesis is structured as follows.
In Chapter \ref{chapter:background}, we provide the necessary background information on blockchain, smart contracts, smart contract security, Algorand, fuzzing and previous related work.
Moving to Chapter \ref{chapter:methodology}, we outline the methodology behind AlgoFuzz's development, discussing our design choices, the features and constraints that influenced these decisions, and conclude with a relevant case study.
Chapter \ref{chapter:evaluation} lays out the research questions, sets up the experimental environment, and presents the results of the experiments.
These results are then discussed in Chapter \ref{chapter:discussion} in the context of the research questions. Furthermore, in the same chapter, we discuss the limitations of our work by provided possible threats to validity.
Finally, in Chapter \ref{chapter:conclusion}, we conclude the thesis by summarizing our findings and discussing possible future work.



