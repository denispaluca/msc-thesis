% !TeX root = ../main.tex
% Add the above to each chapter to make compiling the PDF easier in some editors.

\chapter{Introduction}\label{chapter:introduction}
In recent years, blockchain technology has gained immense traction, revolutionizing various industries because of its decentralized and transparent nature.
The rapid growth of blockchain technology has brought a new era of decentralized applications, which are made possible by smart contracts.
Smart contracts are one of the key components of most modern blockchain networks.
They are self-executing programs that run on the blockchain and are used to automate the execution of agreements between two or more parties.
As smart contracts handle an increasing set of sensitive transactions, ranging from financial agreements to supply chain management, making them secure has become a top priority.
Coupled with the fact that smart contracts are immutable, it makes exploiting vulnerabilities in smart contracts a lucrative target for attackers.
In the recent past, we have witnessed several high-profile attacks on smart contracts, which have resulted in the loss of billions of dollars in value.

As with other software, different software testing techniques have been proposed and are being actively used to detect vulnerabilities in smart contracts.
Traditionally software testing methodologies such as unit testing, have shown to be minimally effective in improving the security of smart contracts when used on their own.
Unit tests are only as good as the test cases written by the developer, and it is impossible to write test cases for all possible inputs.
For this reason, fuzzing has been proposed as a complementary technique to unit testing.

Fuzz testing or fuzzing is a software testing technique that involves providing invalid, unexpected, random or semi-random data as inputs to a program.




